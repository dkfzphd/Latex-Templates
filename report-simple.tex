\documentclass[a4paper,twoside]{report}
\usepackage{ctex}%中文
\usepackage[margin=1in]{geometry}
\usepackage[colorlinks=true,linkcolor={black},breaklinks,citecolor={black}]{hyperref}
% 文章内链接
\usepackage{cite} %简介连续引用的问题[4][5][6]变成[4-7]
\setlength\parindent{2em}%段落缩进 首自然段不影响
\setlength\parskip{1em}%段间距
\usepackage{amsfonts, amsmath, amssymb}
\usepackage[dvipsnames]{xcolor}
\usepackage{upgreek}%roman正立upright 希腊字母
\usepackage{siunitx}% 标准物理单位
\usepackage{chemfig}%化学结构画图
\usepackage[most]{tcolorbox}% text tcolorbox package
\tcbset{colback=black!5!white,colframe=black!65!white} % globe set tcolorbox
\usepackage{zhlipsum}
\usepackage{lastpage}
\usepackage{fancyhdr} % for use of \pageref{LastPage}
\fancypagestyle{IHA-fancy-style}{%
  \fancyhf{}% Clear header and footer
  \fancyhead[LE,RO]{\slshape \rightmark}
  \fancyhead[LO,RE]{\slshape \leftmark}
  \fancyfoot[C]{\thepage\ of \pageref{LastPage}}% Custom footer
  \renewcommand{\headrulewidth}{0.4pt}% Line at the header visible
  \renewcommand{\footrulewidth}{0.4pt}% Line at the footer visible
}

% Redefine the plain page style
\fancypagestyle{plain}{%
  \fancyhf{}%
  \fancyfoot[C]{\thepage\ of \pageref{LastPage}}%
  \renewcommand{\headrulewidth}{0pt}% Line at the header invisible
  \renewcommand{\footrulewidth}{0.4pt}% Line at the footer visible
}
\title{A short example}
\author{good author}
\begin{document}

\maketitle
\tableofcontents
\pagestyle{IHA-fancy-style}

\chapter{A chapter}
\zhlipsum[1]

\newpage

\section{section1}
Look at the footer
\zhlipsum[2-6]
\subsection{section1}
Look at the footer
\zhlipsum[2-6]
\section{section2}
Look at the footer
\zhlipsum[2-6]
\section{section3}
Look at the footer
\zhlipsum[2-6]


\chapter{B chapter}
\zhlipsum[1]

\newpage

\section{section1}
Look at the footer
\zhlipsum[2-6]
\section{section2}
Look at the footer
\zhlipsum[2-6]
\section{section3}
Look at the footer
\zhlipsum[2-6]

\chapter{C chapter}
\zhlipsum[1]

\newpage

\section{section1}
Look at the footer
\zhlipsum[2-6]
\section{section2}
Look at the footer
\zhlipsum[2-6]
\section{section3}
Look at the footer
\zhlipsum[2-6]

\begin{tcolorbox}[title=\textbf{$\ggg$ My nice heading}]
This  is  another  \textbf{tcolorbox}.
\tcblower
Here,  you  see  the  lower  part  of  the  box.
\end{tcolorbox}

\begin{tcolorbox}[colback=white!95!red,colframe=black!30!red,title=\textbf{$\ggg$ My  nice  heading}]
$\ggg$ This  is  another with sets  \textbf{tcolorbox}.
English news is a new education format. You will find articles and news from all over the world. English news is a school newspaper.
\tcblower
Here,  you  see  the  lower  part  of  the  box.
\end{tcolorbox}

\section{Math and other sth IMPORTANT!}
中文的引\cite{RN865}``\textbf{测试}", "测试", "English", ``english"\\
BamHI, \textit{m} = $\upmu$g \textit{m} = 5 kg 不建议直接输入会导致分开 μg
%$\upmu$g μg 单位要用正立的希腊字母 物理量的代表要用斜体 
\SI{44.9}{\celsius}, \SI{44.9}{\celsius} \SI{}{\ohm}, 100 $\Omega$\\
because $\because$ therefore $\therefore$

\begin{equation}
\frac{m_1}{vol_1}=\frac{m_2}{vol_2}
\label{density}
\end{equation}
我们使用公式\eqref{density}可以计算任一缺失参数。% 公式的交叉引用
$5\times6 = 30$\\
$6\div5 = 1.2$\\
Bar plot should be displayed with $mean \pm sd$ \par%plus minus symbol \pm

\begin{equation}
\sin^2(x)+\cos^2(x) =1
\end{equation}

$\underbrace{\overbrace{(a+b+c)}^6
\cdot  \overbrace{(d+e+f)}^7}
_\text{meaning  of  life}  =  42$

$$\underbrace{\overbrace{(a+b+c)}^6
\cdot  \overbrace{(d+e+f)}^7}
_\text{meaning  of  life}  =  42$$\\
$\underbrace{that\quad is\quad life!}_\text{命运啊,生活啊}$

\begin{equation}
2CO + O_2\xrightarrow[burning]{fire}  2CO_2  
\label{CO+O2}
\end{equation}

公式\eqref{CO+O2}是$CO$燃烧反应方程式
\subsection{subsection one}
\zhlipsum[1-2]

\subsubsection{subsubsection one}
\zhlipsum[1-2]		
\newpage

\section{Math symbol}
This is our second document.6 $\mu m$ 
\chemfig{A-[:50]B-[:-25]C}
Branched molecule \vspace{.5cm}

\chemfig{H-C(-[2]H)(-[6]H)-C(=[1]O)-[7]H}
\quad\chemfig{A-[:50]B-[:-25]C}\vspace{1cm}\\

{ 
    \setchemfig{atom sep=2em,bond style={line width=1pt,red,dash pattern=on 2pt off 2pt}}  
    \chemname
    {\chemfig{H-C(-[2]H)(-[6]H)-C(=[1]O)-[7]H}}    
    {Ethanal}
}\vspace{.5cm}
\begin{figure}[!h]
\centering
\includegraphics[width=0.7\textwidth]{Rplot}
\caption{This is a plot of sin($x$)}
\end{figure}

\begin{figure}[!htb]
\minipage{0.32\textwidth}
  \includegraphics[width=\linewidth]{Rplot}
  \caption{A really Awesome Image}\label{fig:awesome_image1}
\endminipage\hfill
\minipage{0.32\textwidth}
  \includegraphics[width=\linewidth]{Rplot}
  \caption{A really Awesome Image}\label{fig:awesome_image2}
\endminipage\hfill
\minipage{0.32\textwidth}%
  \includegraphics[width=\linewidth]{Rplot}
  \caption{A really Awesome Image}\label{fig:awesome_image3}
\endminipage
\end{figure}


$\underbrace{a\Rightarrow b}$
\section{Statement of the Problem}
\textit{This is our second document.}\cite{RN853} It contains a title and a 
section with text.tourists visit London every year\cite{RN869}. The capital city of England and the United Kingdom attracts a lot of visitors, especially Americans. This is probably due to the position it holds in the world economy.
Not by chance London has been defined by some of the most important sociologists of the XX century as a global city. That’s how people and cultures blend together continuously, creating a multicultural place, where 1/3 of the population is foreign-born. Being a language student, I was impressed when I discovered that over 200 different idioms are spoken in London.
In my travel itineraries I usually try to focus on some unusual places to visit, because I find them more fascinating than visiting the typical monuments or museums, those which everybody already knows. Thus, here you are my top 5 sites in London
\subsection{Statement2}
\zhlipsum[2-8]
\subsubsection{Statement22}
\zhlipsum[2-8]
\subsection{Statement3}
\zhlipsum[2-8]
\subsubsection{Statement33}
\zhlipsum[2-8]

\newpage
\section{Hypothesis}
This is our second document. It contains a title and a 
section with text.tourists visit London every year. The capital city of England and the United Kingdom attracts a lot of visitors, especially Americans. This is probably due to the position it holds in the world economy.
Not by chance London has been defined by some of the most important sociologists of the XX century as a global city. That’s how people and cultures blend together continuously, creating a multicultural place, where 1/3 of the population is foreign-born. Being a language student, I was impressed when I discovered that over 200 different idioms are spoken in London.
In my travel itineraries I usually try to focus on some unusual places to visit, because I find them more fascinating than visiting the typical monuments or museums, those which everybody already knows. Thus, here you are my top 5 sites in London
\subsection{StatementfffProblem}
\zhlipsum[2-9]
\newpage
\section{Materials}
This is our second document. It contains a title and a 
section with text.tourists visit London every year. The capital city of England and the United Kingdom attracts a lot of visitors, especially Americans. This is probably due to the position it holds in the world economy.
Not by chance London has been defined by some of the most important sociologists of the XX century as a global city. That’s how people and cultures blend together continuously, creating a multicultural place, where 1/3 of the population is foreign-born. Being a language student, I was impressed when I discovered that over 200 different idioms are spoken in London.
In my travel itineraries I usually try to focus on some unusual places to visit, because I find them more fascinating than visiting the typical monuments or museums, those which everybody already knows. Thus, here you are my top 5 sites in London

\newpage
\section{Procedure}
This is our second document. It contains a title and a 
section with text.tourists visit London every year. The capital city of England and the United Kingdom attracts a lot of visitors, especially Americans. This is probably due to the position it holds in the world economy.
Not by chance London has been defined by some of the most important sociologists of the XX century as a global city. That’s how people and cultures blend together continuously, creating a multicultural place, where 1/3 of the population is foreign-born. Being a language student, I was impressed when I discovered that over 200 different idioms are spoken in London.
In my travel itineraries I usually try to focus on some unusual places to visit, because I find them more fascinating than visiting the typical monuments or museums, those which everybody already knows. Thus, here you are my top 5 sites in London
\newpage
\section{Result}
This is our second document. It contains a title and a 
section with text.tourists visit London every year. The capital city of England and the United Kingdom attracts a lot of visitors, especially Americans. This is probably due to the position it holds in the world economy.
Not by chance London has been defined by some of the most important sociologists of the XX century as a global city. That’s how people and cultures blend together continuously, creating a multicultural place, where 1/3 of the population is foreign-born. Being a language student, I was impressed when I discovered that over 200 different idioms are spoken in London.
In my travel itineraries I usually try to focus on some unusual places to visit, because I find them more fascinating than visiting the typical monuments or museums, those which everybody already knows. Thus, here you are my top 5 sites in London
\newpage
\section{Conclusion}
This is our second document. It contains a title and a 
section with text.tourists visit London every year. The capital city of England and the United Kingdom attracts a lot of visitors, especially Americans. This is probably due to the position it holds in the world economy.\par
Not by chance London has been defined by some of the most important sociologists of the XX century as a global city. That’s how people and cultures blend together continuously, creating a multicultural place, where 1/3 of the population is foreign-born. Being a language student, I was impressed when I discovered that over 200 different idioms are spoken in London.\par
In my travel itineraries I usually try to focus on some unusual places to visit, because I find them more fascinating than visiting the typical monuments or museums, those which everybody already knows. Thus, here you are my top 5 sites in London
\newpage
\section{Comment}
This is our second document\cite{RN872,RN853,RN863,RN871}. It contains a title and a 
section with text.tourists visit London every year. The capital city of England and the United Kingdom attracts a lot of visitors, especially Americans. This is probably due to the position it holds in the world economy.
Not by chance London has \cite{RN942} been defined by some of the most important sociologists of the XX century as a global city. That’s how people and cultures blend together continuously, creating a multicultural place, where $\frac{1}{3}$ of the population is foreign-born. Being a language student, I was impressed when I discovered that over 200 different idioms are spoken in London.
In my travel itineraries I usually try to focus on some unusual places to visit, because I find them more fascinating than visiting the typical monuments or museums, those which everybody already knows. Thus, here you are my top 5 sites in London

{\setlength{\parindent}{0pt}

\bibliographystyle{unsrt}
\bibliography{citation}
}
\thispagestyle{empty}
\end{document}
